% !TEX TS-program = pdflatexmk-c
\documentclass[11pt, letterpaper]{article}

\usepackage[brazilian, english]{babel}
\usepackage[utf8]{inputenc}
\usepackage[T1]{fontenc}
\usepackage[draft]{todonotes}   % notes showed
\usepackage{geometry}
\usepackage{lipsum}
\usepackage{enumitem}
\usepackage{float}
\usepackage{forest}


% set custom date and format
\usepackage[nodayofweek,level]{datetime}
\newcommand{\mydate}{\formatdate{11}{3}{2016}}
\newdateformat{mydateformat}{ \THEMONTH \dateseparator \THEDAY \dateseparator \THEYEAR}

\title{\vspace{-2cm} 19702 \\ Project 1 (First Draft)}
\author{Francisco Fonseca \and Octavio Mesner}
\date{\mydate}

\date{\mydateformat\normalsize\mydate} % Today's date or a custom date

\begin{document}

\maketitle % Print the title

\section{Introduction} \label{intro}

New York City (NYC) would like to implement a fleet of autonomous
vehicles (AV) by 2020 as part of a larger initiative of establishing
itself as a world leader in smart city infrastructure. Because of the
novel nature of AV technology, there is a considerable amount of
uncertainty regarding the impact from its implementation on traffic
issues including
safety, congestion, energy consumption and environmental impacts.
City officials wish to determine the effect of each alternative, while
considering uncertainty, on these outcome
and optimal strategies moving forward with AV
technology within the NYC fleet of vehicles. This study will focus on
the alternatives available to implement  this technology
from AutoMerge, Inc. (AM) specifically in NYC's 40-passenger transit
buses. It will compare different alternative strategies by performing
a Benefit Cost Analysis (BCA) to assess how these different options
impact the general population of NYC, which are the main stakeholders
in this issue. Among the factors to be included in the analysis are:
safety, energy consumption, air pollution, GHG emissions, weather
conditions and traffic flow. Section \ref{problem} presents a detailed
description of the problem and the alternatives analyzed. Section
\ref{results} presents the analysis of each alternative and the
results found. In section \ref{sensitivity} a sensitivity analysis is
performed for some inputs. Section \ref{discussion} discusses the
results of the analyses and section \ref{conclusion} presents the
conclusion and recommendations of this study.

\section{Problem Description} \label{problem}
\subsection{Alternatives}

Broadly, NYC must consider the following three alternatives.

\begin{description}%[leftmargin=0pt]
\item[Alternative 1:] Do not implement AV.

This is the reference alternative. AV is not implemented and the
benefits and costs are the ones already incurred to the population.
There is relatively little uncertanty associated with this outcome
because there is historical data to project the effect of this
alternative on outcomes.

\item[Alternative 2:] Implement AV in the NYC bus fleet.

In this alternative, the city implements the AV technology in the bus
fleet directly (without performing any pilot tests).  Because this is
an emerging technology, it is unclear how AV will perform in each
outcome so uncertainty analysis will help determine the expected value
for each outcome using the risk information currently available.

\item[Alternative 3:] Perform a pilot test with an amount of $n$ buses
  before deciding to implement AV.

In this alternative, the city performs a pilot test with a predefined
amount of $n$ buses. The pilot test has an associated cost directly
proportional to $n$.  The potential benefit of running a pilot study
is the additional information gained on risks associated with
congestion and other outcomes.  It will be necessary to calculate the
value of imperfect information for each $n$ for this alternative.


\end{description}

\subsection{Benefits, Costs \& Uncertainty}

Table \ref{tab:bca.costs} presents the different direct and indirect
costs associated with the implementation of AV technology in NYC bus
fleet.  Each cost may be slightly different under different
circumstances.  In our analysis, we assume uncertainty with respect to
weather and AM proformance.

\begin{table}[h]
\caption{Costs associated with implementing the AV technology in NYC buses}
\begin{center}
\begin{tabular}{l}
\hline\hline
Capital costs of installing AM system in buses\\
O \& M costs of AM system \\
Cost of building simulator facility to train drivers \\
Traffic fatalities \\
Traffic injuries \\
Traffic Congestion \\
Emission of air pollutants \\
Emission of Greenhouse gases (GHG) \\
Cost of potential pilot test (*) \\
\hline\hline
{\footnotesize (*) only applicable to alternative 2}
\end{tabular}
\end{center}
\label{tab:bca.costs}
\end{table}%

\subsection{Assumptions}

\begin{enumerate}
\item The number of riders using each mode of transportation will remain
  stable for all three alternatives.
\item Changes in congestion will affect motor vehicles (buses, cars,
  light duty vehicles, and trucks) but not bicycles or walkers.
\item All travel minutes across persons and modes of transportations
  are worth the same and includes fuel costs, \$22 per hour per individual.
\item Weather patterns remain consistent.
\item Any travel time, even fast ones, incurs a time cost.  This means
  that for each alternative, there will be a time cost added which is
  associated with it.
\end{enumerate}

\section{Methods}

For this analysis, we use a decision tree with standard expected
values to assess alternatives one and two and the value of imperfect
information to assess alternative 3.  Figure~\ref{decisiontree} shows
our decision tree for the first two alternatives.  This same paradigm
will be used to compute the expected value of imperfect information.

\begin{figure}
\centering
\begin{forest}
for tree={grow=east}
[Decision
   [Alt 1[7,tier=word]]
   [Alt 2
      [Good Weather
         [MB[1,tier=word]][LB[2,tier=word]][LW[3,tier=word]]
      ]
      [Bad Weather
         [MB[4,tier=word]][LB[5,tier=word]][LW[6,tier=word]]
      ]
   ]
]
\end{forest}
\caption{Partial decision tree}\label{decisiontree}
\end{figure}

\begin{enumerate}
\item \textbf{Congestion.}  To calculate congestion costs, we only
  consider
  buses, cars, light duty vehicles, and trucks.  While walkers and
  cyclists do experience travel times, we do not believe that they
  will change with the different alternatives and do not need to be
  included for comparison purposes.  Thus, for any single mode of
  transportation, we calculate cost as follows:
  \[\mbox{Cost}=\mbox{daily person trips}\times\mbox{average trip
    hours}\times\mbox{Cost per hour per individual}.\]
  From our assumptions above, only average trip hours will change
  under differing alternatives, weather conditions, and AM
  performance.


\item \textbf{Emissions.}  Calculating emmissions costs are similar to
  congestion cost because they depend on average trip
  hours$\times$daily person trips but each mode of transportation has
  differing  costs associated due to differing emissions.
  \todo{determine these costs}

\item \textbf{Fatalities \& Injuries.}  There are several ways to
  estimate fatalities and injuries since mean rates are given per
  trip, exposure time, distance traveled.  Since congestion and
  emissions are already calculated in person/trip-time, we will
  continue this for fatalities and injuries for consistency.

\item \textbf{Capital Costs.}
  Since the cost of buses will be incurred in all alternatives, we
  only consider the additional cost for AM which only affects
  alternatives two and three.
\end{enumerate}

\section{Analysis and Results} \label{results}

\todo[inline]{Write results.}

\subsection{Alternative 1: Don't Implement AV}

\subsection{Alternative 2: Pilot test}

\subsection{Alternative 3: Implement AV}

\section{Sensitivity Analysis} \label{sensitivity}



\todo[inline]{Write sensitivity analysis.}

\section{Discussion} \label{discussion}

\todo[inline]{Write discussion.}

\section{Conclusion \& Recommendations} \label{conclusion}

\todo[inline]{Write Conclusion and Recommendations.}


\end{document}
