% !TEX TS-program = pdflatexmk-c
\documentclass[11pt, letterpaper]{article}

\usepackage[brazilian, english]{babel}
\usepackage[utf8]{inputenc}
\usepackage[T1]{fontenc}

\usepackage[draft]{todonotes}   % notes showed

\usepackage{geometry}
\usepackage{lipsum}

\usepackage{enumitem}

\usepackage{float}

% set custom date and format
\usepackage[nodayofweek,level]{datetime}
\newcommand{\mydate}{\formatdate{11}{3}{2016}}
\newdateformat{mydateformat}{ \THEMONTH \dateseparator \THEDAY \dateseparator \THEYEAR}

\title{\vspace{-2cm} 19702 \\ Project 1 (First Draft)}
\author{Francisco Fonseca \and Octavio Mesner}
\date{\mydate}

\date{\mydateformat\normalsize\mydate} % Today's date or a custom date

\begin{document}

\maketitle % Print the title

\section{Introduction} \label{intro}

Hello!\\

New York City (NYC) would like to implement a fleet of autonomous
vehicles (AV) by 2020 as part of a larger initiative of establishing
itself as a world leader in smart city infrastructure. Because of the
still novel nature of AV technology, there is a great amount of
uncertainty on how its implementation will affect traffic issues, such
as safety, congestion and environmental impacts. In such context, city
officials wish to understand which alternatives they should consider
and how they should proceed in terms of strategies in regards to AV
technology within the NYC fleet of vehicles. This study will focus on
the alternatives available to implement an automatic merging system
from AutoMerge, Inc. (AM) specifically in NYC's 40-passenger transit
buses. It will compare different alternative strategies by performing
a Benefit Cost Analysis (BCA) to assess how these different options
impact the general population of NYC, which are the main stakeholders
in this issue. Among the factors to be included in the analysis are:
safety, energy consumption, air pollution, GHG emissions, weather
conditions and traffic flow. Section \ref{problem} presents a detailed
description of the problem and the alternatives analyzed. Section
\ref{results} presents the analysis of each alternative and the
results found. In section \ref{sensitivity} a sensitivity analysis is
performed for some inputs. Section \ref{discussion} discusses the
results of the analyses and section \ref{conclusion} presents the
conclusion and recommendations of this study.

\section{Problem Description} \label{problem}

\subsection{Benefits \& Costs}

Table \ref{tab:bca.costs} presents the different direct and indirect
costs associated with the implementation of AV technology in NYC bus
fleet.

\begin{table}[h]
\caption{Costs associated with implementing the AV technology in NYC buses}
\begin{center}
\begin{tabular}{l}
\hline\hline
Capital costs of installing AM system in buses\\
O \& M costs of AM system \\
Cost of building simulator facility to train drivers \\
Traffic fatalities \\
Traffic injuries \\
Traffic Congestion \\
Emission of air pollutants \\
Emission of Greenhouse gases (GHG) \\
Cost of potential pilot test (*) \\
\hline\hline
{\footnotesize (*) only applicable to alternative 2}
\end{tabular}
\end{center}
\label{tab:bca.costs}
\end{table}%

\subsection{Assumptions}

\todo[inline]{Write about assumptions.}

\subsection{Alternatives}

Three different alternatives will be analyzed in this study. The following list presents these three alternatives:

\begin{description}%[leftmargin=0pt]
\item[Alternative 1:] Do not implement AV.

This is the reference alternative. AV is not implemented and the benefits and costs are the ones already incurred to the population.

\item[Alternative 2:] Perform a pilot test with an amount of $n$ buses before deciding to implement AV.

In this alternative, the city performs a pilot test with a predefined amount of $n$ buses. The pilot test has an associated fixed cost, but its brings more information about the actual benefits of AV on traffic congestion.

\item[Alternative 3:] Implement AV in the NYC bus fleet.

In this alternative, the city implements the AV technology in the bus fleet directly (without performing any pilot tests).

\end{description}

\section{Analysis and Results} \label{results}

\todo[inline]{Write results.}

\subsection{Alternative 1: Don't Implement AV}

\subsection{Alternative 2: Pilot test}

\subsection{Alternative 3: Implement AV}

\section{Sensitivity Analysis} \label{sensitivity}

\todo[inline]{Write sensitivity analysis.}

\section{Discussion} \label{discussion}

\todo[inline]{Write discussion.}

\section{Conclusion \& Recommendations} \label{conclusion}

\todo[inline]{Write Conclusion and Recommendations.}


\end{document}
